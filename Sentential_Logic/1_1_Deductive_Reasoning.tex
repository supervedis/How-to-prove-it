\documentclass{article}
\title{Deductive Reasoning and Logical Connectives}
\usepackage{amsmath}
\usepackage{amssymb}
\begin{document}
\textbf{Exercises}
\begin{itemize}
    \item[*1.] Analyze the logical forms of the following statements:
    \begin{itemize}
        \item[(a)] We'll have either a reading assignment or homework problems, but we won't have both homework problems and a test.
        \item[] R - Reading assignment.
        \item[] H - Homework problems.
        \item[] T - Test.
        \item[] $(R \lor H)\land \neg(H \land T)$
        \item[(b)] You won't go skiing, or you will and there won't be any snow.
        \item[(c)] $\sqrt{7} \nleqslant 2$.
    \end{itemize}
    \item[2.] Analyze the logical forms of the following statements:
    \item[3.] Analyze the logical forms of the following statements:
    \item[4.] Analyze the logical forms of the following statements:
    \item[5.] Which of the following expressions are well-formed formulas?
    \item[*6.] Let P stand for the statement “I will buy the pants” and S for the statement “I will buy the shirt.” What English sentences are represented by the following formulas?
    \item[7.] Let S stand for the statement “Steve is happy” and G for “George is happy.” What English sentences are represented by the following formulas?
    \item[8.] Let T stand for the statement “Taxes will go up” and D for “The deficit will go up.” What English sentences are represented by the following formulas?
    \item[9.] Identify the premises and conclusions of the following deductive arguments and analyze their logical forms. Do you think the reasoning is valid? (Although you will have only your intuition to guide you in answering this last question, in the next section we will develop some techniques for determining the validity of arguments.)
\end{itemize}
\end{document}